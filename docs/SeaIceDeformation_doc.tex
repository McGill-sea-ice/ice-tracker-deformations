\documentclass[12pt]{article}
\usepackage{graphicx, array, gensymb, amsmath, amssymb, enumitem, subcaption, siunitx, float, dirtree, geometry}

\geometry{margin=0.72in}

\graphicspath{{./}}

\newcommand{\ode}[2]{\frac{d #1}{d #2}}

\begin{document}

\title{%
  Deformations from Sea-Ice Tracker Data \\
  \large Lekima Yakuden, Béatrice Duval}

\section*{Introduction}

This documentation focuses on the use of the data-set-generating and data-set-analyzing scripts, as the structure of the entire repository and the theory behind the computations. The document is organised as follows:

\begin{enumerate}
    \item Usage
    \item Structure
    \item Methodology
\end{enumerate}

The Usage section will introduce the user-editable options followed by step-by-step instructions on how to run the scripts. The Structure section will discuss the structure of the repository and includes descriptions of all relevant files. The Methodology section will discuss the equations and methods used to compute the deformations.

\section{Usage}\label{Usage}

The user must first install and activate the MiniConda environment before running any of the functions. The Conda environment (dependencies) can be installed and activated by navigating to the project's home directory (\textbf{ice-tracker-deformations}) and executing the following two lines:

\begin{verbatim}
    $ conda env create -f environment.yaml
    $ conda activate icetrackdefs
\end{verbatim}



\subsection{Data-set Generation}

Only two files, \textbf{namelist.ini} and \textbf{main.py} must be edited or ran by the user to generate deformation data sets. The user-editable parameters in \textbf{namelist.ini} control \textbf{main.py}.

\subsubsection*{\textbf{namelist.ini}}

The deformation calculation code's settings are editable in \textbf{namelist.ini}. The structure of the file is as follows:

\subsubsection*{[IO]}

\begin{itemize}
    \item $data\_folder$: Path to the directory where the raw $.dat$ files are stored. All of the data files must be uncompressed in this directory to be processed.
    \item $output\_folder$: Path to the desired output directory. All functions which save files will create the path if the target directory does not already exist. Figures and NetCDF4 files will be saved to this directory.
    \item $exp$: Name of the current experiment. This is the name of the directory that will be stored at $output\_path$ containing all outputs.
\end{itemize}

\subsubsection*{[Grid]}

This section is depreciated and will be removed in subsequent versions.

\subsubsection*{[Metadata]}

\begin{itemize}
    \item $ice\_tracker$: Name of the satellite used to collect the raw data. Used in making titles for files and figures. (e.g. S1 (for Sentinel-1))
    \item $tracking\_error$: Spatial uncertainty of the $ice\_tracker$ in metres.
\end{itemize}

\subsubsection*{[Processing$\_$options]}

\begin{itemize}
    \item $overwrite$: Boolean (yes / no) to overwrite previous results if they exist under the same output path and experiment name.
    \item $method$: The first of two methods, M00, is being depreciated. It is recommended that method M01 be used (set the variable = M01).
    \item $visualise$: Boolean (yes / no) that determines whether or not the deformations will be visualised along side the NetCDF4 file being generated.
\end{itemize}

\subsubsection*{[Date$\_$options]}

\begin{itemize}
    \item $start\_year$: Start year of analysis. 
    \item $start\_month$: Start month of analysis.
    \item $start\_day$: Start day of analysis.
    \item $end\_year$: End year of analysis.
    \item $end\_month$: End month of analysis.
    \item $end\_day$: End day of analysis.
    \item $timestep$: Timestep (between images) in hours.
    \item $tolerance$: The buffer, in hours, the time step is afforded. (e.g. $tolerance = 3$ will result in a time step of $timestep \pm 3$ hours).
\end{itemize}

\subsubsection*{\textbf{main.py}}

\begin{enumerate}
    \item Set $data\_folder$ and $output\_folder$ to the desired target directories.
    \item Set $exp$ to the desired experiment name.
    \item Set $ice\_tracker$ and $tracking\_error$ to the name of the satellite and its tracking error, respectively.
    \item Set your start and end dates. Zero-pad them according to the ISO format, e.g. 04 instead of 4.
    \item Set $timestep$ and $tolerance$ as desired.
    \item Save changes made to \textbf{namelist.ini}.
\end{enumerate}

Once the options in \textbf{namelist.ini} have been set, all that is left to do is to run \textbf{main.py} by navigating to \textbf{src/SeaIceDeformation} and executing

\begin{verbatim}
    $ python main.py
\end{verbatim}

\subsection{Data-set Analysis}

As discussed in \ref{structure} Data-set analysis, there are two files with analysis functions in them. 

\textbf{coverage$\_$frequency$\_$map.py} is designed to analyse \textit{raw data files} (features tracked from pairs of satellite images). \textbf{netcdf$\_$tools.py} is designed to analyze and edit \textit{NetCDF4 files} produced by the deformation computation scripts. These two files can be run in the same manner as all other python files from the terminal:

\begin{verbatim}
    $ python coverage_frequency_map.py
\end{verbatim}

or 

\begin{verbatim}
    $ python netcdf_tools.py
\end{verbatim}

Both of these files have customizable options, found in \textbf{options.ini}. This one $.ini$ file controls all tools.

\subsubsection*{\textbf{options.ini}}
This files contains the following editable parameters:

\subsubsection*{[IO]}

\begin{itemize}
    \item $data\_folder$: Path to the directory containing raw (.dat) data files. All of the data files must be uncompressed in this directory to be processed. Used only in $coverage\_frequency\_map.py$.
    
    \item $output\_folder$: Path to the desired output directory. All functions which save files will create the path if the target directory does not already exist. Figures and NetCDF4 files will be saved to this directory.
    
    \item $netcdf\_path$: Path to the NetCDF4 file to be staged for reading or editing. Used only in $netcdf\_tools.py$. This is the direct path to the NetCDF file, not the directory which contains it. This path should end with $.nc$
\end{itemize}

\subsubsection*{[meta]}

\begin{itemize}
    \item $ice\_tracker$: Name of the satellite used to collect the raw data. Used in making titles for files and figures. (e.g. S1 (for Sentinel-1))
\end{itemize}

\subsubsection*{[options]}

The date options are applied to tools in both $coverage\_frequency\_map.py$ and $netcdf\_tools.py$
\begin{itemize}
    \item $start\_year$: Start year of analysis. 
    \item $start\_month$: Start month of analysis.
    \item $start\_day$: Start day of analysis.
    \item $end\_year$: End year of analysis.
    \item $end\_month$: End month of analysis.
    \item $end\_day$: End day of analysis.
    \item $timestep$: Timestep (between images) in hours. Note that when plotting from a NetCDF file, this timestep must be set accurately to generate the correct titles.
    \item $tolerance$: The buffer, in hours, the time step is afforded. (e.g. $tolerance = 3$ will result in a time step of $timestep \pm 3$ hours). This option only affects $coverage\_frequency\_map.py$.
    \item $interval$: The interval used to imitate model outputs in $coverage\_frequency\_map.py$, in hours. Must be larger or equal to $timestep$, as multiple time steps can be taken within an interval (e.g. three 24 hour time steps in a 72 hour interval).
    \item $resolution$: Cell size of heat maps generated by $coverage\_frequency\_map.py$ in kilometres.
    \item $area\_filter$: Boolean to turn the area filtering function in $netcdf\_tools.py$ on or off. If True, the NetCDF will be filtered to only include data in a circle set by the next three parameters.
    \item $centre\_lat$: Latitude of the centre point of the masking circle.
    \item $centre\_lon$: Longitude of the centre point of the masking circle.
    \item $radius$: Radius of the masking circle.
\end{itemize}

Functions are turned on and off by setting the variables corresponding to the functions in \textbf{options.ini} to True and False, respectively.

\subsection*{Tools}

\subsubsection*{\textbf{coverage\_frequency\_map.py}}

\subsubsection*{$visualise\_coverage\_histogram2d$}

This function is on by default when running \textbf{coverage\_frequency\_map.py}. To generate a 2-D histogram of coverage:

\begin{enumerate}
    \item Set $data\_folder$ and $output\_folder$ to the desired target directories.
    \item Set $ice\_tracker$ to the name of the satellite(s) used to capture the raw image pairs. Keep it as one string, e.g. RCMS1 or RCM$\_$S1 but not RCM S1.
    \item Set your start and end dates. Zero-pad them according to the ISO format, e.g. 04 instead of 4.
    \item Set $timestep$, $tolerance$, and $resolution$ as desired.
    \item Ensure that $visualise\_timeseries$ and $visualise\_interval$ are set to False.
    \item Save \textbf{options.ini}
    \item Run \textbf{coverage\_frequency\_map.py}.
\end{enumerate}

\subsubsection*{$visualise\_timeseries$}

This function must be turned on in \textbf{options.ini} by setting $visualise\_timeseries$ to True. To plot a timeseries of spatial coverage:

\begin{enumerate}
    \item Set $data\_folder$ and $output\_folder$ to the desired target directories.
    \item Set $ice\_tracker$ to the name of the satellite(s) used to capture the raw image pairs. Keep it as one string, e.g. RCMS1 or RCM$\_$S1 but not RCM S1.
    \item Set your start and end dates. Zero-pad them according to the ISO format, e.g. 04 instead of 4.
    \item Set $timestep$, $tolerance$, $interval$, and $resolution$ as desired. Note that the $interval$ must be greater than or equal to the $timestep$.
    \item Set $visualise\_timeseries$ to True.
    \item Save \textbf{options.ini}
    \item Run \textbf{coverage\_frequency\_map.py}.
\end{enumerate}

\subsubsection*{$interval\_frequency\_histogram$}

This function must be turned on in \textbf{options.ini} by setting $interval\_frequency\_histogram$ to True. To plot a heat map representing interval-based coverage:

\begin{enumerate}
    \item Set $data\_folder$ and $output\_folder$ to the desired target directories.
    \item Set $ice\_tracker$ to the name of the satellite(s) used to capture the raw image pairs. Keep it as one string, e.g. RCMS1 or RCM$\_$S1 but not RCM S1.
    \item Set your start and end dates. Zero-pad them according to the ISO format, e.g. 04 instead of 4.
    \item Set $timestep$, $tolerance$, $interval$, and $resolution$ as desired. Note that the $interval$ must be greater than or equal to the $timestep$.
    \item Set $visualise\_interval$ to True.
    \item Save \textbf{options.ini}
    \item Run \textbf{coverage\_frequency\_map.py}.
\end{enumerate}

\subsubsection*{\textbf{netcdf\_tools.py}}

Functions in this file, designed to read and edit NETCDF4 files, are mainly controlled by user-editable masks in space and time. All functions must take a date range, set by the user in \textbf{options.ini}, as input. If the dates selected by the user do not correspond to the availability of data in the NetCDF, an error will be thrown. Optionally, the user can also create a circular area mask by setting $area_filter$ to True and setting the centre point's coordinates (in WGS 84, Lat/Lon) and setting a value for the radius. If $area_filter$ is set to False, all data available in the specified time range will be processed.

\subsubsection*{$plot\_start\_end\_points$}

This function must be turned on in \textbf{options.ini} by setting $plot\_start\_end\_points$ to True. To plot start and end points of the loaded NetCDF's triangle vertices:

\begin{enumerate}
    \item Set $output\_folder$ and $netcdf\_path$ to the desired target directory and file path.
    \item Set your start and end dates. Zero-pad them according to the ISO format, e.g. 04 instead of 4.
    \item Set $plot\_start\_end\_points$ to True.
    \item Save \textbf{options.ini}
    \item Run \textbf{netcdf\_tools.py}.
\end{enumerate}

\subsubsection*{$plot\_deformations$}

This function must be turned on in \textbf{options.ini} by setting $plot\_deformations$ to True. To plot deformations from the loaded NetCDF:

\begin{enumerate}
    \item Set $output\_folder$ and $netcdf\_path$ to the desired target directory and file path.
    \item Set your start and end dates. Zero-pad them according to the ISO format, e.g. 04 instead of 4.
    \item Set $timestep$ to the time step of the NetCDF file. This must be done to accurately title the plots.
    \item Set $plot\_deformations$ to True.
    \item Save \textbf{options.ini}
    \item Run \textbf{netcdf\_tools.py}.
\end{enumerate}

\subsubsection*{$write\_netcdf$}

This function must be turned on in \textbf{options.ini} by setting $write\_netcdf$ to True. To write and save a new NetCDF by splicing the loaded NetCDF:

\begin{enumerate}
    \item Set $output\_folder$ and $netcdf\_path$ to the desired target directory and file path.
    \item Set your start and end dates. Zero-pad them according to the ISO format, e.g. 04 instead of 4.
    \item Set $timestep$ to the time step of the NetCDF file. This must be done to accurately title the output NetCDF.
    \item Set $write\_netcdf$ to True.
    \item Save \textbf{options.ini}
    \item Run \textbf{netcdf\_tools.py}
\end{enumerate}

\section{Structure}\label{Structure}

The project is structed like so within \textbf{ice-tracker-deformations}'s \textbf{src} directory:

\begin{figure}[H]
    \centering
    \begin{minipage}[b]{.5\textwidth}
        \dirtree{%
            .1 .
            .2 SatelliteCoverage.
            .3 config.py.
            .3 coverage$\_$frequency$\_$map.py.
            .3 netcdf$\_$tools.py.
            .3 options.ini.
        }
    \end{minipage}%
    \begin{minipage}{.5\textwidth}
        \dirtree{%
            .1 .
            .2 SeaIceDeformation.
            .3 \textit{M00$\_$d00$\_$crop$\_$grid.py}.
            .3 \textit{M00$\_$d01$\_$delaunay$\_$triangulation.py}.
            .3 \textit{M00$\_$d02$\_$to$\_$grid$\_$coord$\_$system.py}.
            .3 \textit{M00$\_$d03$\_$compute$\_$deformations.py}.
            .3 M01$\_$d01$\_$delaunay$\_$triangulation.py.
            .3 M01$\_$d03$\_$compute$\_$deformations.py.
            .3 config.py.
            .3 main.py.
            .3 namelist.ini.
            .3 utils$\_$datetime.py.
            .3 utils$\_$deformation$\_$computations.py.
            .3 utils$\_$get$\_$data$\_$paths.py.
            .3 utils$\_$grid$\_$coord$\_$system.py.
            .3 utils$\_$load$\_$data.py.
            .3 utils$\_$load$\_$grid.py.
            .3 utils$\_$visualise$\_$grid.py.
            .3 visualise$\_$deformation.py.
        }
    \end{minipage}
\end{figure}

Files in \textit{italics} (\textit{M00$\_$*}) are depreciated and may not be present in future versions. The project runs using the MiniConda virtual environment discussed in \ref{Usage} Usage.

\subsection{Data-set Generation}

Files and functions designed to compute and visualise deformations from raw data files (image pairs) exist in the \textbf{SeaIceDeformation} subdirectory. The raw data files, containing a cloud of points, are formed into a mesh using a Delaunay triangulation which is then used to compute deformations according to line integral approximations (sums around each triangle based on strain rates). These data are then stored in a NETCDF4 file, and optionally plotted on a basemap. Only \textbf{namelist.ini} must be edited before generating deformation data sets. Directories with .csv files corresponding to each processed data file will be generated to be used in intermediate steps of the process, and will remain stored in the desired output path.

\begin{itemize}
    \item \textbf{M01$\_$d01$\_$delaunay$\_$triangulation.py}: Module that performs a Delaunay triangulation on a set of X/Y data points for all raw DAT files listed in config. The results are stored in a triangulated CSV file for each data file, using the triangulated data paths listed in config.
    \item \textbf{M01$\_$d03$\_$compute$\_$deformations.py}: Module that computes sea-ice deformations for triangle cells for all triangulated CSV files listed in config. The results are then
    stored in a calculations CSV file for each data file, in addition to a single output NetCDF file that combines the results from all processed datasets, using the calculations and output data paths listed in config.
    \item \textbf{config.py}: This file contains helper functions for processing user inputs and data.
    \item \textbf{main.py}: This file is the main file to be executed to perform data set generation.
    \item \textbf{utils$\_$datetime.py}: Module that provides tools for finding the starting and ending times of a data file as Datetime objects given its filename, and for finding the time lapse in seconds between two Datetime objects.
    \item \textbf{utils$\_$deformation$\_$computations.py}: This file contains helper functions for deformation computations, used in \textbf{M01$\_$d03$\_$compute$\_$deformations.py}.
    \item \textbf{utils$\_$get$\_$data$\_$paths.py}: Module that provides tools that produce data file paths for each stage of data processing given a raw filename, an output folder, an experience name, a starting date and a duration (all of which are parameters in namelist.ini). These data file paths are used to write and load data files from all stages of data processing.
    \item \textbf{utils$\_$grid$\_$coord$\_$system.py}: Module that provides tools for the conversion of a triangle’s starting and ending positions from Latitude/Longitude to X/Y coordinates in a local Cartesian grid coordinate system.
    \item \textbf{utils$\_$load$\_$data.py}: This file contains helper functions for loading raw data files.
    \item \textbf{utils$\_$load$\_$grid.py}: This file contains helper functions for loading the RIOPS grid.
    \item \textbf{utils$\_$visualise$\_$grid.py}: This file contains functions for plotting information on the RIOPS grid.
    \item \textbf{visualise$\_$deformation.py}: This file contains functions to visualise deformation data on a basemap.
\end{itemize}

\subsubsection*{NetCDF output}

The output NETCDF4 file contains the following meta data:

\begin{itemize}
    \item Tracking error: Satellite's resolution.
    \item Satellite name.
    \item Reference time: 00:00:00 on the first date of data availability.
\end{itemize}

Each row of the NetCDF represents a triangle (produced by the Delaunay triangulation) formed by three vertices, the start and end positions in latitude/longitude of which take up six columns (three start positions and three end positions). Each row (triangle)) is further populated by the start and end times (seconds after the reference time corresponding to the start/end positions), deformation values (divergence, shear, and vorticity), the triangulation information (taking up seven columns: id1, id2, id3, idtri, id$\_$start$\_$lat1, id$\_$start$\_$lat2, id$\_$start$\_$lat3), and the strain rates ($\ode{u}{x}$, $\ode{u}{y}$, $\ode{v}{x}$, $\ode{v}{y}$). All deformation and strain rate values are in units of $1/days$.

\subsection{Data-set Analysis}\label{structure}

Scripts for analysing the data sets (in NETCDF4 format) and raw data exist in the \textbf{SatelliteCoverage} subdirectory. 

\begin{itemize}
    \item \textbf{config.py}: This file contains helper functions to load, sort, and filter raw satellite image pairs (henceforce called data files), as well as a function to read the user's preferences from \textbf{options.ini}.
    
    \item \textbf{coverage$\_$frequency$\_$map.py}: This file contains functions designed to analyze the spatio-temporal coverage of raw data files.

        \begin{itemize}
            \item \textbf{visualise$\_$coverage$\_$histogram2d}: This function takes in x and y points in the \textit{NSIDC Sea Ice Polar Stereographic North} (EPSG 3413) projection representing the tracked features, the maximum and minimum dates of the data's time range, and the user-set timestep. It then plots a 2-D histogram at the user-specified resolution, where the value of some bin (cell) is the number of data points which landed in that bin during the specified time range, at that timestep. The histogram is constructed using NumPy's \textit{histogram2d} function.
            
            \item \textbf{coverage$\_$histogram2d}: This is a helper function for \textbf{coverage$\_$timeseries} and \textbf{interval} frequency histogram2d. Like \textbf{visualise$\_$coverage$\_$histogram2d}, this function bins points from the raw data files into a 2-D histogram to investigate coverge, but outputs a binary, 2-D array, where cells with \underline{any} data have values of 1 and cells with no data have values of 0.
            
            \item \textbf{coverage$\_$timeseries}: This function plots a timeseries of coverage (in $\%$ of the Arctic Ocean) from raw data files. The area of the Arctic Ocean, as well as the size of the grid cells that cover it, can be modified in the source code. 
            
            \item \textbf{interval$\_$frequency$\_$histogram2d}: Similar to \textbf{visualise$\_$coverage$\_$histogram2d}, this function plots a heatmap of spatio-temporal coverage in chunks (intervals) of time. Each data point associated with a specified \textit{time step} (e.g. 24, 72 hours) that lands in a cell within the specified time \textit{interval} will switch that cell's value from 0 to 1. The purpose of this is to imitate model outputs, which may have a generation frequency of three days (One output every three days). 
        \end{itemize}

    \item \textbf{netcdf$\_$tools.py}: This file contains functions designed to analyse and splice NetCDF files created by the deformation calculation scripts in \textbf{src/SeaIceDeformation} directory.
    
        \begin{itemize}
            \item \textbf{plot$\_$start$\_$end$\_$points}: This function plots the start and end points for the triangle vertices in the loaded NetCDF. This translates to six points, three blue start points and three red end points, per row in the NetCDF. This illustrates the rough coverage area of the NetCDF.
            
            \item \textbf{recreate$\_$coordinates}: This function is necessary in plotting the deformations from the NetCDF file. The triangulation information that is embedded in the NetCDF when running the deformation calculation scripts (\textbf{src/SeaIceDeformation}) requires that the vertices (lat/lon points) of each of the triangles be in the order they were processed in during the intial loading of the raw data files. This function takes an ID associated with each triangle vertex (in the NetCDF file as \textbf{start$\_$lat$\_$id1, 2, 3}) and effectively recreates the conditions under which the deformations were calculating, allowing for accurate plotting.
            
            \item \textbf{plot$\_$deformations}:This function  plots the deformation information stored in the 
            
            
            NetCDF file. It saves three plots to the user's specified output folder: one illustrating divergence, one illustrating vorticity, and one representing shear. It uses MatPlotLib's \textit{ax.tripcolor} function to plot and colour triangles according to the same colourbars used in the original deformation visualisation (located in \textbf{src/SeaIceDeformation}).
        
        \end{itemize}

    \item \textbf{options.ini}: This file contains the user-editable options for all of the above scripts. It is broken down into five section: IO, meta, options, coverage$\_$frequency, and netcdf$\_$tools. The parameters in this file will be read by the generation and analysis scripts. This file is also where the user can specify which tools to use.
\end{itemize}

\section{Methodology}

This section will discuss the methodology behind the deformation calculations.

\subsection{Computing Deformations}

We compute deformations for every triangle using X/Y coordinates in a local Cartesian coordinate system as computed in the previous step and following \textit{Bouchat et al. (2020)}. The following is taken from \textit{Bouchat et al. (2020)} and presents the equations used to compute deformations.


%%%%%%%%%%%%%%%%%%%%
The total sea-ice deformation rates are defined as:
\begin{equation}
\dot{\epsilon}_{tot} = \left[ \dot{\epsilon}^2_{I}+\dot{\epsilon}^2_{II} \right]^{1/2} \;,
\end{equation}
where 
\begin{eqnarray}
\dot{\epsilon}_{I} &= &\frac{\partial u}{\partial x}+\frac{\partial v}{\partial y} \, , \label{eq:eps_I} \\
\dot{\epsilon}_{II} &= &\left[ \left( \frac{\partial u}{\partial x} - \frac{\partial v}{\partial y}\right)^{2} + \left( \frac{\partial u}{\partial y} + \frac{\partial v}{\partial x}\right)^{2}   \right]^{1/2} \, , \label{eq:eps_II} 
\end{eqnarray}
are the strain rate invariants (or divergence and maximum shear strain rate respectively), and $(u,v)$ are the ice velocity components. In a Lagrangian framework, the velocity derivatives (or strain rates) are most easily obtained using line integrals around the cells contour as per the divergence theorem \textit{(Lindsay, 2003)}: 
\begin{eqnarray}
\frac{\partial u}{\partial x}  = \frac{1}{A}\oint u \textrm{d}y \, ,&  \hspace{0.4cm} & \frac{\partial u}{\partial y}  = -\frac{1}{A}\oint u \textrm{d}x \, , \nonumber \\
\frac{\partial v}{\partial x}  = \frac{1}{A}\oint v \textrm{d}y \, ,&  \hspace{0.4cm}  & \frac{\partial v}{\partial y}  = -\frac{1}{A}\oint v \textrm{d}x\, ,
\label{eq:dudx}
\end{eqnarray}
where A is the Lagrangian cell area. If $(x_{i}, y_{i})$ is the position of the cell vertex $i$ ($=1,2,..,n$; counterclockwise) at time $t$ and $(u_{i}, v_{i})$ its velocity during the time interval $\Delta t_{i}$, we can approximate the line integrals using the trapezoidal rule as:
\begin{equation}
\frac{\partial u}{\partial x}  = \frac{1}{A}\sum_{i=1}^{n}\frac{1}{2}\left( u_{i+1} + u_{i} \right)\left( y_{i+1} - y_{i} \right) \, ,
\label{eq:dudx_sum}
\end{equation}
 where
 \begin{equation}
 A = \frac{1}{2}\sum_{i=1}^{n}\left( x_{i}y_{i+1} -  x_{i+1}y_{i} \right)\, ,
\label{eq:A}
 \end{equation}
and $x_{n+1} = x_1$, and similar other cyclical equalities for $y_{n+1}$, $u_{n+1}$, and $v_{n+1}$. Similar equations also hold for the other velocity derivatives. Here, $(u_{i},v_{i})$ is computed as $(\Delta x_{i}/\Delta t_{i}, \Delta y_{i}/\Delta t_{i})$, where $\Delta x_{i}$ and $\Delta y_{i}$ are the displacement of the cell's vertices during the time interval $\Delta t_{i}$. Note that Equation (\ref{eq:A}) is also derived using the divergence theorem and the trapezoidal rule.
%%%%%%%%%%%%%%%%%%%%%

\end{document}